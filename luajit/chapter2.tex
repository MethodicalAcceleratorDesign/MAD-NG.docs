\chapter{Virtual Machine}
\label{Chapt:VM}
  
%===============================================================================
%% Frontend %
%===============================================================================

\section{Frontend}
\label{Sec:frontend}

%============================================================
% Lexer
%============================================================

\subsection{Lexer}
\label{Subsec:Lexer}

The lexer is implemented in the \emph{lj\_lex.c} and \emph{lj\_lex.h} files.
It use \emph{LexState} as a principal data structure. The user-provided
\emph{rfunc} function is used to read a chunk of data to process. It is accessed
through the \emph{p} and \emph{pe} pointers. The main function is
\emph{lex\_scan} that dispatches the work to other functions depending of the
type of data to be processed (comment, string literal, long string, numbers
etc...). TValues (\emph{tokval}, \emph{lookaheadval}) are used to store the
token values were \emph{LexToken} (\emph{tok}, \emph{lookahead}) determine the
type. The string buffer (\emph{sb}) is used to accumulate characters of a future
string before internalizing it. All Lua keyword are internalized as a string at
the very beginning, GCstr has the field \emph{reserved} for marking them.

\begin{lstlisting}[style=CStyle]
typedef struct LexState {
  struct FuncState *fs; /* Current FuncState (lj_parse.c). */
  struct lua_State *L;  /* Lua state. */
  TValue tokval;        /* Current token value. */
  TValue lookaheadval;  /* Lookahead token value. */
  const char *p;        /* Current position in input buffer. */
  const char *pe;       /* End of input buffer. */
  LexChar c;            /* Current character. */
  LexToken tok;         /* Current token. */
  LexToken lookahead;   /* Lookahead token. */
  SBuf sb;              /* String buffer for tokens. */
  lua_Reader rfunc;     /* Reader callback. */
  void *rdata;          /* Reader callback data. */
  BCLine linenumber;    /* Input line counter. */
  BCLine lastline;      /* Line of last token. */
  GCstr *chunkname;     /* Current chunk name (interned string). */
  const char *chunkarg; /* Chunk name argument. */
  const char *mode;     /* load bytecode (b), source text (t). */
  VarInfo *vstack;      /* Stack for local variables. */
  MSize sizevstack;     /* Size of variable stack. */
  MSize vtop;           /* Top of variable stack. */
  BCInsLine *bcstack;   /* Stack for bytecode instr./line numbers. */
  MSize sizebcstack;    /* Size of bytecode stack. */
  uint32_t level;       /* Syntactical nesting level. */
} LexState;
\end{lstlisting}

%============================================================% Parser
%============================================================

\subsection{Parser}
\label{Subsec:Parser}

The parser of LuaJIT is implemented in \emph{lj\_parse.c} and \emph{lj\_parse.h}
files. It also uses \emph{LexState} as the principal data-structure. it is
responsible for calling the lexer to get tokens. LuaJIT doesn't really have an
abstract syntax tree representation of the parsed code. Instead it directly
generates the bytecode on the fly using helpers from \emph{lj\_bc.h}.
The \emph{lj\_parse} function is the entry point and parse the main chunck as a
vararg function. The unit of emission is the function (\emph{GCproto}) and the structure used for the construction is \emph{FuncState}. Parsing is a succession
of chuncks (\emph{parse\_chunk}) were \emph{parse\_stmt} is the principal
function called for each line, dispatching the work depending of the current
token type. \emph{FuncScope} is a linked-list of structure used for scope
management.

\begin{lstlisting}[style=CStyle]
typedef struct FuncState {
  GCtab *kt;              /* Hash table for constants. */
  LexState *ls;           /* Lexer state. */
  lua_State *L;           /* Lua state. */
  FuncScope *bl;          /* Current scope. */
  struct FuncState *prev; /* Enclosing function. */
  BCPos pc;               /* Next bytecode position. */
  BCPos lasttarget;       /* BCsb position of last jump target. */
  BCPos jpc;              /* Pending jump list to next bytecode. */
  BCReg freereg;          /* First free register. */
  BCReg nactvar;          /* Number of active local variables. */
  BCReg nkn, nkgc;        /* Number of lua_Number/GCobj constants */
  BCLine linedefined;     /* First line of function definition. */
  BCInsLine *bcbase;      /* Base of bytecode stack. */
  BCPos bclim;            /* Limit of bytecode stack. */
  MSize vbase;            /* Base of variable stack for this func. */
  uint8_t flags;          /* Prototype flags. */
  uint8_t numparams;      /* Number of parameters. */
  uint8_t framesize;      /* Fixed frame size. */
  uint8_t nuv;            /* Number of upvalues */
  VarIndex varmap[...];   /* Map from register to variable idx. */
  VarIndex uvmap[...];    /* Map from upvalue to variable idx. */
  VarIndex uvtmp[...];    /* Temporary upvalue map. */
} FuncState;
\end{lstlisting}

%============================================================
% Bytecode frontend
%============================================================

\subsection{Bytecode frontend}
\label{Subsec:bc-frontend}

Another frontend feature provided by LuaJIT is the possibility to save and load
bytecode directly, allowing to skip the lexer and parser phase.

The writing part is handled by the module \emph{bcsave.lua} that use the
\emph{lj\_bcwrite} function from \emph{lj\_bcwrite.c} to generate the data to be written.

The reading part is done by the code in \emph{lj\_bcread.c} file. When it is
detected that the input file is a bc dump instead of a plain Lua code,
\emph{cpparser} from \emph{lj\_load.c} calls \emph{lj\_bcread} instead of
\emph{lj\_parse} normally. This reader also use \emph{LexState} as the principal
data-structure.

%===============================================================================
% Internals
%===============================================================================

 \section{Internals}
 \label{Sec:Internals}
    
%====================================================================
% Tagged value
%====================================================================

\subsection{Tagged value}
\label{Subsec:tagged-value}

LuaJIT represent all internal elements as 64-bits \emph{TValue} (tagged value).
It uses the nan-tagging technique to differentiate between numbers and other
types of element. In fact, \emph{lua\_number} are 64-bits floating-points numbers
following the \emph{ieee} standard. Doing so, numeric nan are canonized by the
cpu FPU (0xfff8 in msb and zero otherwise), letting the possibility to use the
lower bits to represent arbitrary data. Internally, LuaJIT has two different
representations, one for 32-bits and another for 64-bits (\emph{LJ\_GC64}) mode
(see Tables \ref{tab:tagged-gc32} and \ref{tab:tagged-gc64}). In those tables,
\emph{itypes} are numbers identifying the type of the object.
GC object (Garbage-Collected Object) represent all allocated object
that are managed by the garbage collector. \emph{GCRef} are references to such
object.

\begin{table}[H]
\centering
\caption{Internal object tagging for 32-bits mode}
\label{tab:tagged-gc32}
\begin{tabular}{l|c|c|c|}
\cline{2-4}
                                              & \multicolumn{2}{c|}{MSW}         & LSW         \\ \hline
\multicolumn{1}{|l|}{size}                    & \multicolumn{2}{c|}{32-bits}     & 32-bits     \\ \hline
\multicolumn{1}{|l|}{primitive types}         & \multicolumn{2}{c|}{itypes}      &             \\
\multicolumn{1}{|l|}{lightuserdata (32-bits)} & \multicolumn{2}{c|}{itypes}      & void *      \\
\multicolumn{1}{|l|}{lightuserdata (64-bits)} & 0xffff           & \multicolumn{2}{c|}{void *} \\
\multicolumn{1}{|l|}{GC objects}              & \multicolumn{2}{c|}{itypes}      & GCRef       \\
\multicolumn{1}{|l|}{int}                     & \multicolumn{2}{c|}{itypes}      & int         \\
\multicolumn{1}{|l|}{number}                  & \multicolumn{3}{c|}{double}                    \\ \hline
\end{tabular}
\end{table}

\begin{table}[H]
\centering
\caption{Internal object tagging for 64-bits mode}
\label{tab:tagged-gc64}
\begin{tabular}{l|c|c|c|c|}
\cline{2-5}
                                      & \multicolumn{3}{c|}{MSW}         & LSW         \\ \hline
\multicolumn{1}{|l|}{size}            & 13-bits & 4-bits & 15-bits       & 32-bits     \\ \hline
\multicolumn{1}{|l|}{primitive types} & 1...1   & itype  & \multicolumn{2}{c|}{1...1}  \\
\multicolumn{1}{|l|}{lightuserdata}   & 1...1   & itype  & \multicolumn{2}{c|}{void *} \\
\multicolumn{1}{|l|}{GC objects}      & 1...1   & itype  & \multicolumn{2}{c|}{GCRef}  \\
\multicolumn{1}{|l|}{int}             & 1...1   & itype  & 0...0         & int         \\
\multicolumn{1}{|l|}{number}          & \multicolumn{4}{c|}{double}                    \\ \hline
\end{tabular}
\end{table}

%====================================================================
% String Internalization
%====================================================================

\subsection{String Internalization}
\label{Subsec:string-inter}

All strings that are manipulated by LuaJIT are internalized. This includes, all
the strings literals of the user Lua code, all identifiers and tokens of the Lua
language itself and also all the strings used internally by LuaJIT. Internalization
mechanism does that, only one copy of a specific string is kept in memory. If
multiple copies of a same string is requested, a pointer to the internalized
version of the string is returned, instead of doing a new string allocation.
Strings need to be immutable and are null-terminated. String's function are
implemented in the \emph{lj\_str.c} file and internalization is done by the
\emph{lj\_str\_new} function.

For that, it implements a hash table and use a very sparse hash
function. Collisions are handled by the use of singly-chained linked list.
The table is resized and all string rehashed when a 100\% load is reached.
The necessary states are saved in the \emph{global\_State} structure in the
\emph{lj\_obj.h} file.

\begin{lstlisting}[style=CStyle]
typedef struct global_State {
  GCRef *strhash; /* String hash table (hash chain anchors). */
  MSize strmask;  /* String hash mask (size of hash table - 1). */
  MSize strnum;   /* Number of strings in hash table. */
  [...]
}
\end{lstlisting}

%====================================================================% Lua table
%====================================================================

\subsection{Lua table}
\label{Subsec:table}

Tables are garbage-collected objects represented by the structure below
(\emph{GCtab} in \emph{lj\_obj.h}). Functions to manipulate them are in
\emph{lj\_tab.c} and \emph{lj\_tab.h} files. It is composed of an array part and
a hash part. If the array part is small, it is allocated directly after the
structure in memory (collocation functionality), otherwise it is separated. The
hash part is a hash table used to store all non-integer key (or integer too big
to fit in the array part). It is implemented has an array using singly-linked
list for collision, where nodes of the linked list are within the array (not
allocated) and a variation of \emph{Brent's hashing methods} is used.
New integer keys that are bigger than the array part are always inserted
in the hash part until this one is full. It then triggers the resizing of the
table. The new \emph{asize} and \emph{hmask} are both power of 2. The
new \emph{asize} value corresponds to the biggest power of 2 such that at least
50 percent of the integers below it are used as key. The new \emph{hmask} is
picked such that all non-integer keys plus the integer keys that are bigger than
the new \emph{asize} fit in it. When a resizing occur the hash values are
re-hashed and integer key that does fit now in the array part are reintroduced
there.

The \emph{nomm} field of GCtab is a
negative cache for fast metamethods checks. It is a bitmap marking absent fields
of the metatable.

\begin{lstlisting}[style=CStyle]
typedef struct GCtab {
  GCHeader;
  uint8_t nomm;    /* Negative cache for fast metamethods. */
  int8_t colo;     /* Array colocation
  (number of slot directly following in memory).*/
  MRef array;      /* Array part. */
  GCRef gclist;
  GCRef metatable; /* Must be at same offset in GCudata. */
  MRef node;       /* Hash part. */
  uint32_t asize;  /* Size of array part (keys [0, asize-1]). */
  uint32_t hmask;  /* Hash part mask (size of hash part - 1). */
#if LJ_GC64
  MRef freetop;    /* Top of free elements. */
#endif
} GCtab;
\end{lstlisting}

%====================================================================
% Garbage collector
%====================================================================

\subsection{Garbage collector}
\label{Subsec:gc}

LuaJIT garbage collector has currently a tricolor, incremental mark and sweep
type implementation. You can find a presentation of it on the wiki
\cite{luajit-gc} in the tri-color section. Its source code can be found in the
\emph{lj\_gc.h} and \emph{lj\_gc.c} files. It uses the GCState structure has a
principal structure. The \emph{gc\_onestep} function is the principal one as it
implements the states machine of the gc. Principal states with their meaning can be
found in the enum below.

\begin{lstlisting}[style=CStyle]
enum {
  GCSpause,
  GCSpropagate,   /* One gray object is processed. */
  GCSatomic,      /* Atomic transition from mark to sweep phase. */
  GCSsweepstring, /* Sweep one chain of strings from the table. */
  GCSsweep,       /* Sweep a few object from root. */
  GCSfinalize     /* Finalize one userdata or cdata object. */
};
\end{lstlisting}

\begin{lstlisting}[style=CStyle]
typedef struct GCState {
  GCSize total;         /* Memory currently allocated. */
  GCSize threshold;     /* Memory threshold. */
  uint8_t currentwhite; /* Current white color. */
  uint8_t state;        /* GC state. */
  uint8_t nocdatafin;   /* No cdata finalizer called. */
  uint8_t unused2;
  MSize sweepstr;       /* Sweep position in string table. */
  GCRef root;           /* List of all collectable objects. */
  MRef sweep;           /* Sweep position in root list. */
  GCRef gray;           /* List of gray objects. */
  GCRef grayagain;      /* List of objects for atomic traversal. */
  GCRef weak;           /* List of weak tables (to be cleared). */
  GCRef mmudata;        /* List of userdata (to be finalized). */
  GCSize debt;          /* Debt (how much GC is behind schedule). */
  GCSize estimate;      /* Estimate of memory actually in use. */
  MSize stepmul;        /* Incremental GC step granularity. */
  MSize pause;          /* Pause between successive GC cycles. */
} GCState;
\end{lstlisting}

%====================================================================
% Allocator
%====================================================================

\subsection{Allocator}
\label{Subsec:alloc}

LuaJIT has its own embedded allocator that is a customized version of \emph{dlmalloc}
(Doug Lea's Malloc). Information on the original implementation can be found on
the web article \cite{dlmalloc-art} or in the comment of the code
\cite{dlmalloc-impl}. The allocator is implemented in the \emph{lj\_alloc.c}
file. Its main structure \emph{malloc\_state} is shown below. Memory on the
heap are allocated in chunks. Free chunks are managed as double linked-list with
the size of the chunk at the beginning and end of it. Unallocated memory are
grouped in bins of corresponding size. There are two types of bins. The
smaller one contains chunk of same size and the top are anchored in
\emph{smallbins}. The bigger one are stored as bitwise digital trees (aka tries)
keyed by size were the top of a tree is anchored in \emph{treebins}. The
allocator differentiates with two types of memory allocation, if it is higher than
128KB then it asks the operating system for a new memory segment using mmap, if
it is lower than that it uses chunks from the current segment. All allocated
segments are kept in a linked list anchored in \emph{seg}. For such smaller
allocation the allocator first try to find an exact fit from the available
chunks to optimize for internal fragmentation. If it cannot find one and that
the requested size is smaller than \emph{dvsize} then it uses the \emph{dv}
chunk (designated victim) which is the last chunk that has been split. It does
so to optimize for locality. Otherwise it goes for a best-fit match. If no chunk
big enough is available, it asks the system to extend the segment and use the
boundary chunk \emph{top} (always kept free). When memory is freed it does chunk
coalescing to avoid memory fragmentation. If \emph{topsize} is bigger than
\emph{trim\_check}, then the current segment is shrinked and the memory is given
back to the OS. \emph{release\_checks} is a decreasing counter that when it
riches zero triggers a check of all segments to release empty ones back to the OS.

\begin{lstlisting}[style=CStyle]
struct malloc_state {
  binmap_t  smallmap;       /* marking non-empty smallbins */
  binmap_t  treemap;        /* marking non-empty treebins  */
  size_t    dvsize;         /* designated victim size      */
  size_t    topsize;        /* boundary chunk size         */
  mchunkptr dv;             /* designated victim           */
  mchunkptr top;            /* boundary chunk              */
  size_t    trim_check;     /* threshold size to release   */
  size_t    release_checks; /* counter for mmap release    */
  mchunkptr smallbins[...]; /* anchor for top small bins   */
  tbinptr   treebins[...];  /* anchor for top tree bins    */
  msegment  seg;            /* anchor for chained segments */
};
\end{lstlisting}

%====================================================================
% Function
%====================================================================

\subsection{Function}
\label{Subsec:func}

There are two distinct representation for function, the function's prototype,
and the function's closure. Lua functions' prototypes are represented by \emph{GCproto} (\emph{lj\_obj.h}), and are followed by the functions' bytecodes in memory.
The closures are represented by the GCfuncL for Lua function and GCfuncC for
c function (using Lua's api). They contain the necessary information for
upvalues. Upvalues are represented by the \emph{GCupval} that if close, contains
the corresponding value or a reference to the stack slot containing the
appropriate value otherwise. Closure can be managed by the functions present in
\emph{lj\_func.c} allowing to create them, destroy them and closing their
upvalues.

\begin{lstlisting}[style=CStyle]
typedef struct GCproto {
  GCHeader;
  uint8_t numparams; /* Number of parameters. */
  uint8_t framesize; /* Fixed frame size. */
  MSize sizebc;      /* Number of bytecode instructions. */
  [...]
  GCRef gclist;
  MRef k;            /* Split constant array (ptr to the middle).*/
  MRef uv;           /* Upvalue list. local slot or parent uv idx. */
  MSize sizekgc;     /* Number of collectable constants. */
  MSize sizekn;      /* Number of lua_Number constants. */
  MSize sizept;      /* Total size including colocated arrays. */
  uint8_t sizeuv;    /* Number of upvalues. */
  uint8_t flags;     /* Miscellaneous flags. */
  uint16_t trace;    /* Anchor for chain of root traces. */
  [...]
} GCproto;
\end{lstlisting}

%====================================================================
% Fast Function
%====================================================================

\subsection{Fast Function}
\label{Subsec:ffunc}

Fast functions are specially optimized standard library function. There are two
parts for each function, the implementation part and the fallback handler.
The implementation part is done in assembly in the \emph{vm\_*.dasc} files and
handle the general case, called the fast pass. If the fast pass fail, the vm
call the corresponding fallback handler (all \emph{LJLIB\_ASM} marked functions)
that will try to recover from the failure if possible. Examples of recoverable
cases might be : a wrong argument type if coercion succeeds or a stack overflow
if stack reallocation succeeds (see comment in lj\_lib.h).

%====================================================================
% GC64 mode
%====================================================================

\subsection{GC64 mode}
\label{Subsec:gc64}

By default LuaJIT use 32-bit pointers which limit its memory on x86 to 4 GB on a
32-bit platform and even 2 GB on a 64-bit platform (due to constant pointers being
embedded as offsets in x86 addressing modes, which are signed 32-bit integers).
This limitation has been there for many years but recently a \emph{GC64} mode
has been developed and is currently in beta, allowing to fully use the well-known
47-bit address space. To activate this mode you just have to uncomment in
\emph{src/makefile} the LUAJIT\_ENABLE\_GC64 variable and to perform a full
recompilation of LuaJIT.


%===============================================================================
% Bytecode interpreter
%===============================================================================

\section{Bytecode interpreter}
\label{Sec:BI}

Most of the VM mechanism is written using the DynASM syntax. To get a better
understanding of the underlining technology see Appendix \ref{Apendix:DynASM}.

The low-level VM code written for each architecture using this assembler can be
found in \emph{vm\_(target).dasc} files. It is mainly composed of the switch
case that generates the assembly code for each byte code, but it also contains
code for other part that requires low level implementation such as the
stack unwinding or the implementation of the fast path of asm fast library functions
(see section \ref{Subsec:ffunc} for more).


%===============================================================================
% Library
%===============================================================================

\section{Library}
\label{Sec:Library}
    
%!TEX root = ../../../FYP_Dissertation.tex

%====================================================================
% Standard library
%====================================================================

\subsection{Standard library}
\label{Subsec:std-lib}

LuaJIT is providing a full compatibility with Lua 5.1 and to do so, it
implements the standard library. The code is copied and adapted from the
\emph{PUC-RIO} Lua interpreter. A list of the corresponding files and the
descriptions of what they are doing can be found in table
\ref{tab:library-std-files}.

\begin{table}[H]
\centering
\caption{Implementation's files of the Lua standard library}
\label{tab:library-std-files}
\begin{tabular}{|l|l|}
\hline
\multicolumn{1}{|c|}{File name} & \multicolumn{1}{c|}{Description}                     \\ \hline
lib\_base.c                     & Base and coroutine library.                          \\
lib\_debug.c                    & Debug library.                                       \\
lib\_init.c                     & Load and initialize standard libraries.              \\
lib\_io.c                       & Files and I/O library.                               \\
lib\_math.c                     & Math library (abs, sqrt, log, random, etc...).       \\
lib\_os.c                       & OS library (date, time, execute, remove, etc...).    \\
lib\_package.c                  & Package library (load, require, etc...).             \\
lib\_string.c                   & String library (gsub, match, etc...).                \\
lib\_table.c                    & Table library (new, clear, insert, foreach, etc...). \\ \hline
\end{tabular}
\end{table}

%====================================================================
% LuaJIT extensions
%====================================================================

\subsection{LuaJIT extensions}
\label{Subsec:lj-extensions}

In addition to the standard library, LuaJIT comes equipped with some library
extensions \cite{extensions}. In addition of a few improvements of existing
modules it provides three new extension modules, namely, \emph{bit} in
\emph{lib\_bit.c} that provides with bitwise operations \cite{bitOp}, \emph{ffi}
in \emph{lib\_ffi.c} that provides functions to interact with the FFI library
(see FFI Chapter \ref{Sec:FFI} for more) and \emph{JIT} in \emph{lib\_jit.c}
that provides functions allowing to control the behavior of the JIT compiler
engine (see JIT Part \ref{Chapt:JIT}).

%====================================================================
% The C API
%====================================================================

\subsection{The C API}
\label{Subsec:c-api}

All those libraries are implemented using the Lua C API that allows to create
and manipulate Lua data, manage the Lua stack etc... The implementation for those
functions can be found in \emph{lj\_api.c}, \emph{lj\_lib.c} and \emph{lib\_aux.c}.

%====================================================================
% Build Library
%====================================================================

\subsection{Build Library}
\label{Subsec:build-lib}

LuaJIT use multiple tricks to generate files automatically that are included
during compilation to help with the building and loading of the standard library
without the need for manual maintenance. We will describe here the different
steps and what they are useful for.

First of all, if no Lua interpreter
(either PUC-Lua or LuaJIT) is available on the machine a simplified and
minimized one is built from the \emph{minilua.c} file. Then the interpreter is
used to run \emph{genlibbc.lua} that will be responsible for parsing all LuaJIT's
source files searching for the \emph{LJLIB\_LUA} macro that surrounds library
functions name that are written in Lua. It then generates the \emph{buildbm\_libbc.h}
file that contains the Lua bytecodes for all those functions in the
\emph{libbc\_code} array and a mapping of the function name and the bytecodes
offset for that function in \emph{libbc\_map}.

This newly generated file is
built along with all \emph{buildvm\_*} files to create the \emph{buildvm}
program that is used to parse from the library source code all other
\emph{LJLIB\_*} macro and generates some files
(\emph{lj\_bcdef.h, lj\_libdef.h, lj\_ffdef.h, lj\_recdef.h and vmdef.lua})
that will be added to LuaJIT compilation. You can see, in Table
\ref{tab:library-macro} the description of the macros and in Table
\ref{tab:library-generated-files} the description of the corresponding generated
file.

\begin{table}[H]
\centering
\caption{Definition of the macros used to build the library}
\label{tab:library-macro}
\begin{tabularx}{\textwidth}{|l|X|}
\hline
\multicolumn{1}{|c|}{Macro}          & \multicolumn{1}{c|}{Description}                     \\\hline
LJLIB\_MODULE\_*                     & register new module.                                 \\\hline
LJLIB\_CF(name)                      & register C function.                                 \\\hline
LJLIB\_ASM(name)                     & register fast function fallback handler.             \\\hline
LJLIB\_ASM\_(name)                   &
  \begin{tabular}[c]{@{}l@{}}
  register fast function that uses previous\\
  LJLIB\_ASM fallback handler.
  \end{tabular}                                                                             \\\hline
LJLIB\_LUA(name)                     & register Lua function.                               \\\hline
\multirow{4}{*}{LJLIB\_SET(name)}    &
  \begin{tabular}[c]{@{}l@{}}
  register previous Lua stack value into the module\\
  table with \emph{name} has key.
  \end{tabular}
  \begin{itemize}
  \item '!' : last stack value became next function's env
  \end{itemize}                                                                             \\\hline
\multirow{7}{*}{LJLIB\_PUSH(val)}    & push \emph{val} on the Lua stask.
  \begin{itemize}
  \item 'lastcl'  : copy last stack value
  \item 'top-x'   : copy last $x^{th}$ stack value
  \item ' "..." ' : push internalized string
  \end{itemize}                                                                             \\\hline
\multirow{8}{*}{LJLIB\_REC(handler)} & register a handler to record a function.
  \begin{itemize}
  \item '.' : get the function's name
  \item 'name data' :
    \begin{itemize}
      \item \emph{name} of recorder
      \item auxiliary \emph{data} to put in \emph{recff\_idmap}
    \end{itemize}
  \end{itemize}                                                                             \\\hline
LJLIB\_NOREGUV                       & to not register this function in module.             \\\hline
LJLIB\_NOREG                         & to not register a function in lj\_lib\_cf\_*.        \\\hline
\end{tabularx}
\end{table}

\begin{table}[H]
\centering
\caption{Generated files description}
\label{tab:library-generated-files}
\begin{tabularx}{\textwidth}{|l|X|}
\hline
\multicolumn{1}{|c|}{File}          & \multicolumn{1}{c|}{Description}                      \\\hline
\emph{lj\_bcdef.h}                  &
  for each fast functions, \emph{lj\_bc\_ofs} contains the offset from
  \emph{lj\_vm\_asm\_begin} (in \emph{lj\_vm.h}) to the mcode of the function
  and \emph{lj\_bc\_mode} contains the byte code operande mode (all set to
  \emph{BCMODE\_FF}) (see \emph{lj\_bc.h} and this \emph{Introduction} section
  of the wiki \cite{luajit-bc}).                                                            \\\hline
\emph{lj\_ffdef.h}                  & list of all library function name                     \\\hline
\emph{lj\_libdef.h}                 &
  \emph{lj\_lib\_cf\_*} arrays contains the list of function pointers for the
  * library. \emph{lj\_lib\_init\_*} are arrays of packed data describing how
  the corresponding library should be loaded (see \emph{lj\_lib\_register} in
  \emph{lj\_lib.c} for the function that parse those data).                                 \\\hline
\emph{lj\_recdef.h}                 &
  for each library functions \emph{recff\_idmap} contains an optional auxiliary
  data (opcode, literal) allowing to handle similar functionalities in a
  common handler. \emph{recff\_func} contains the list of record handler.                   \\\hline
\emph{vmdef.lua}                    &
  contains all vm definition for use in Lua.
  \begin{itemize}
  \item bcnames  : bytecode names
  \item irnames  : IR instructions names
  \item irfpm    : floating point math function names
  \item irfield  : maps field id to field name (specific field that the IR instruction is accessing)
  \item ircall   : name of special function for calls instruction.
  \item traceerr : maps error num to error message
  \item ffnames  : maps library function id to function name
  \end{itemize}                                                                             \\\hline
\end{tabularx}
\end{table}


%===============================================================================
% Foreign function interface
%===============================================================================

\section{Foreign function interface}
\label{Sec:FFI}

%!TEX root = ../../../FYP_Dissertation.tex

The FFI (Foreign Function Interface) is one of LuaJIT extensions modules. It
allows to call external C functions and use C data structures from pure Lua code.
A very small (not representative) example can be seen below.

\begin{lstlisting}[style=LuaStyle]
local ffi = require("ffi")
ffi.cdef[[
  typedef struct points { int x,y,z; } points;
]]
local p1 = ffi.new("points", { 1, 2, 3 })
print(p1.x) --  1
p1.x = 11
print(p1.x) -- 11
\end{lstlisting}
There is some official documentation for FFI users on \emph{luajit.org}
where you can find the motivation for the FFI module \cite{ffi-motivation},
a small tutorial \cite{ffi-tuto}, the API documentation \cite{ffi-api} and the
FFI semantics \cite{ffi-semantics}. There is also a reflection
library for FFI ctypes \cite{ffi-reflect} and its documentation
\cite{ffi-reflect-doc} for anyone interested in exploring the ctype of a given cdata.

This chapter will present the internal implementation of the FFI and not its use.
The organisation of the information will follow the actual implementation files.

\paratitle{lib\_ffi.c:}\\
This file is the top level file of the FFI library. It contains the
implementation of the FFI API, the function that makes the connection between
Lua and C using the standard Lua \emph{C API}.
It is also responsible for loading the FFI module (\emph{luaopen\_ffi}).
This file mainly use and connect together functionalities implemented in other
files. It is for example responsible to allocate and initialize the main state
(\emph{CTState}) explained below.

\paratitle{lj\_obj.h:}\\
From this file, we are only interested in this chapter by the \emph{GCcdata}
structure that is the garbage-collected object representing any C data use
through and with the FFI. You can see its composition bellow. The main thing to
see here is the \emph{ctypeid} which is the index of the ctype describing the
attached data (the payload follows the structure in memory).
\begin{lstlisting}[style=CStyle]
typedef struct GCcdata {
  GCHeader;
  uint16_t ctypeid;	/* C type ID. */
} GCcdata;
\end{lstlisting}

\paratitle{lj\_ctype.h:}\\
Below is the \emph{CType} data structure responsible for describing to the ffi
what kind of data the \emph{cdata} represent (e.g. variable, struct,
function, etc.). A detailed breakdown of how the information is laid out can
be seen in Table \ref{tab:ffi-ctype}. The meaning of the abbreviations used in
the previous table can be seen in Table \ref{tab:ffi-ctype2}.
\begin{lstlisting}[style=CStyle]
typedef struct CType {
  CTInfo info;   /* Type info. */
  CTSize size;   /* Type size or other info. */
  CTypeID1 sib;  /* Sibling element. */
  CTypeID1 next; /* Next element in hash chain. */
  GCRef name;    /* Element name (GCstr). */
} CType;
\end{lstlisting}

%          +----------------------------+--------+-------+-------+-------+
%          |            info            |        |       |       |       |
%          +----------------------------+  size  |  sid  | next  | name  |
%          |type|  flags | A  |   cid   |        |       |       |       |
% +-------------------------------------+--------------------------------+
% |size    | 4  |    8   | 4  |   16    |   32   |  16   |  16   | GCRef |
% |--------+----------------------------+--------+-------+-------+-------+
% |NUM     |0000|BFcvUL..| A  |    x    | size   |       | type  |       |
% |STRUCT  |0001|..cvu..V| A  |    x    | size   | field | name? | name? |
% |PTR     |0010|..cvR...| A  |   cid   | size   |       | type  |       |
% |ARRAY   |0011|VCcv...V| A  |   cid   | size   |       | type  |       |
% |VOID    |0100|..cv....| A  |    x    | size   |       | type  |       |
% |ENUM    |0101|........| A  |   cid   | size   | const | name? | name? |
% |FUNC    |0110|....VS..|..cc|   cid   | nargs  | field | name? | name? |
% |TYPEDEF |0111|........|....|   cid   |   x    |       | name  | name  |
% |ATTRIB  |1000|....|attrnum |   cid   | attr   | sib?  | type? |       |
% |FIELD   |1001|........|....|   cid   | offset | field |       | name? |
% |BITFIELD|1010|B.cvU| csz   |.bsz|.pos| offset | field |       | name? |
% |CONSTVAL|1011|..c.....|....|   cid   | value  | const | name  | name  |
% |EXTERN  |1100|........|....|   cid   |   x    | sib?  | name  | name  |
% |KW      |1101|........|....|   tok   | size   |       | name  | name  |
% +-------------------------------------+--------+-------+-------+-------+
\begin{table}[p]
\footnotesize
\centering
\caption{Summary of CType informations}
\label{tab:ffi-ctype}
\begin{tabular}{l|c|c|c|c|c|l|c|c|c|c|}
\cline{2-11}
                               & \multicolumn{6}{c|}{info}                                                         & \multirow{2}{*}{size} & \multirow{2}{*}{sid} & \multirow{2}{*}{next} & \multirow{2}{*}{name} \\ \cline{2-7}
                               & type & \multicolumn{2}{c|}{flags}          & A         & \multicolumn{2}{c|}{cid} &                       &                      &                       &                       \\ \hline
\multicolumn{1}{|l|}{size}     & 4    & \multicolumn{2}{c|}{8}              & 4         & \multicolumn{2}{c|}{16}  & 32                    & 16                   & 16                    & GCRef                 \\ \hline
\multicolumn{1}{|l|}{NUM}      & 0000 & \multicolumn{2}{c|}{BFcvUL..}       & A         & \multicolumn{2}{c|}{}    & size                  &                      & type                  &                       \\
\multicolumn{1}{|l|}{STRUCT}   & 0001 & \multicolumn{2}{c|}{..cvu..V}       & A         & \multicolumn{2}{c|}{}    & size                  & field                & name                  & name                  \\
\multicolumn{1}{|l|}{PTR}      & 0010 & \multicolumn{2}{c|}{..cvR...}       & A         & \multicolumn{2}{c|}{cid} & size                  &                      & type                  &                       \\
\multicolumn{1}{|l|}{ARRAY}    & 0011 & \multicolumn{2}{c|}{$V^2$Ccv...V}   & A         & \multicolumn{2}{c|}{cid} & size                  &                      & type                  &                       \\
\multicolumn{1}{|l|}{VOID}     & 0100 & \multicolumn{2}{c|}{..cv....}       & A         & \multicolumn{2}{c|}{}    & size                  &                      & type                  &                       \\
\multicolumn{1}{|l|}{ENUM}     & 0101 & \multicolumn{2}{c|}{........}       & A         & \multicolumn{2}{c|}{cid} & size                  & const                & name                  & name                  \\
\multicolumn{1}{|l|}{FUNC}     & 0110 & \multicolumn{2}{c|}{....$V^3$S..}   & ..cc      & \multicolumn{2}{c|}{cid} & nargs                 & field                & name                  & name                  \\
\multicolumn{1}{|l|}{TYPEDEF}  & 0111 & \multicolumn{2}{c|}{........}       & ....      & \multicolumn{2}{c|}{cid} &                       &                      & name                  & name                  \\
\multicolumn{1}{|l|}{ATTRIB}   & 1000 & ....             & \multicolumn{2}{c|}{attrnum} & \multicolumn{2}{c|}{cid} & attr                  & sib                  & type                  &                       \\
\multicolumn{1}{|l|}{FIELD}    & 1001 & \multicolumn{2}{c|}{........}       & ....      & \multicolumn{2}{c|}{cid} & offset                & field                &                       & name                  \\
\multicolumn{1}{|l|}{BITFIELD} & 1010 & B.cvU            & \multicolumn{2}{c|}{csz}     & .bsz        & .pos       & offset                & field                &                       & name                  \\
\multicolumn{1}{|l|}{CONSTVAL} & 1011 & \multicolumn{2}{c|}{..c.....}       & ....      & \multicolumn{2}{c|}{cid} & value                 & const                & name                  & name                  \\
\multicolumn{1}{|l|}{EXTERN}   & 1100 & \multicolumn{2}{c|}{........}       & ....      & \multicolumn{2}{c|}{cid} &                       & sib                  & name                  & name                  \\
\multicolumn{1}{|l|}{KW}       & 1101 & \multicolumn{2}{c|}{........}       & ....      & \multicolumn{2}{c|}{tok} & size                  &                      & name                  & name                  \\ \hline
\end{tabular}
\end{table}

\begin{table}[p]
\footnotesize
\centering
\caption{Definition of CType fields}
\label{tab:ffi-ctype2}
\begin{tabular}{ll|l|l|}
\hline
\multicolumn{2}{|c|}{\textit{\textbf{flags}}} & \multicolumn{2}{c|}{\textit{\textbf{A}}}                                    \\ \hline
\multicolumn{1}{|l|}{B}      & Boolean        & A                      & allignement of $2^A$ bytes                         \\
\multicolumn{1}{|l|}{F}      & Float          & cc                     & calling convention                                 \\
\multicolumn{1}{|l|}{c}      & const          & attrnum                & Attributes number (see CTA\_* in lj\_ctype.h)      \\
\multicolumn{1}{|l|}{v}      & volatile       & csz                    & size of the memory slot                            \\ \cline{3-4}
\multicolumn{1}{|l|}{U}      & Unsigned       & \multicolumn{2}{c|}{\textit{\textbf{cid}}}                                  \\ \cline{3-4}
\multicolumn{1}{|l|}{L}      & Long           & cid                    & child id                                           \\
\multicolumn{1}{|l|}{u}      & union          & bsz                    & number of bits of the bit-field                    \\
\multicolumn{1}{|l|}{V}      & VLA            & pos                    & starting position in bit inside the memory slot    \\
\multicolumn{1}{|l|}{R}      & reference      & tok                    & token id                                           \\ \cline{3-4}
\multicolumn{1}{|l|}{$V^2$}  & Vector         & \multicolumn{2}{c|}{\textit{\textbf{size}}}                                 \\ \cline{3-4}
\multicolumn{1}{|l|}{C}      & Complex        & size                   & size in bytes                                      \\
\multicolumn{1}{|l|}{$V^3$}  & Vararg         & nargs                  & number of fixed arguments of a function            \\
\multicolumn{1}{|l|}{S}      & SSE arguments  & offset                 & offset in bytes from the start of the struct       \\ \cline{1-2}
                             &                & value                  & the actual constant value                          \\
                             &                & attr                   & value of the attribute                             \\ \cline{3-4}
                             &                & \multicolumn{2}{c|}{\textit{\textbf{sib}}}                                  \\ \cline{3-4}
                             &                & field                  & function args or struc/union field                 \\
                             &                & const                  & constant value                                     \\
                             &                & sib                    & chain of attribute or object of the attribute      \\ \cline{3-4}
                             &                & \textit{\textbf{name}} & GCRef to a string containing the identifier.       \\ \cline{3-4}
                             &                & \textit{\textbf{next}} & chain for hash collisions in cts-\textgreater hash \\ \cline{3-4}
\end{tabular}
\end{table}

The most important struct of the FFI is the \emph{CTState} bellow. It contains
all the internalize ctype in the \emph{tab} table. \emph{finalizer} is a weak
keyed Lua table (values can be garbage collected if the key is not referenced
elsewhere) containing all finalizer registered with the \emph{ffi.gc} method.
\emph{miscmap} is a Lua table mapping all metatable of ctypes registered using
the \emph{ffi.metatype} method in the negative CTypeID range and all callback
functions in the positive callback slot range. Any metatable added to miscmap is
definitive and never collected. \emph{hash} is an array used as a hash table
for quick CTypeID checks. It map both, the hashed name of named elements and the
hashed type (info and size) for unnamed elements to the corresponding CTypeID.
Collisions are handled in a linked list using the \emph{next} field of the
\emph{CType} struct.

\begin{adjustbox}{max width=\textwidth}
\begin{minipage}{\linewidth}
\begin{lstlisting}[style=CStyle]
typedef struct CTState {
  CType *tab;        /* C type table. */
  CTypeID top;       /* Current top of C type table. */
  MSize sizetab;     /* Size of C type table. */
  lua_State *L;      /* Lua state (for errors and allocations). */
  global_State *g;   /* Global state. */
  GCtab *finalizer;  /* Map of cdata to finalizer. */
  GCtab *miscmap;    /* Map -CTypeID->metatable and cb slot->func. */
  CCallback cb;      /* Temporary callback state. */
  CTypeID1 hash[...];/* Hash anchors for C type table. */
} CTState;
\end{lstlisting}
\end{minipage}
\end{adjustbox}

\paratitle{lj\_ctype.c:}\\
This file provides functions to manage \emph{CType}. It is divided into three
parts. The first one is the allocation, creation and internalization of
\emph{CType}. The second one is providing getters to get C type information.
The last one is  type representation, providing the necessary functions to
convert a \emph{CType} to a human readable string representation.
\emph{lj\_ctype\_repr} is the entry function that returns the internalize string
representation. It uses the struct \emph{CTRepr} bellow to create the
representation by appending/prepending characters through the pb/pe pointers
into the buffer. The main function is the \emph{ctype\_repr} that contains a
switch on the \emph{CType} info.

\begin{lstlisting}[style=CStyle]
typedef struct CTRepr {
  char *pb, *pe; /* Points to begining/end inside the buffer*/
  CTState *cts;  /* C type state. */
  lua_State *L;
  int needsp;    /* Next append needs an extra space character */
  int ok;        /* Indicate if buf is currently a valid type */
  char buf[...]; /* String buffer of the ctype being constructed */
} CTRepr;
\end{lstlisting}

\paratitle{lj\_cparse.h:}\\
The cparser is responsible for parsing the string of the C declarations looking
for types or external symbols. Its code structure is quite close to the
Lua lexer/parser. Its principal struct is the \emph{CPState} bellow which is
similar to \emph{LexState}. In this struct, \emph{tmask} is a mask constraining
the possible ctype of the next identifier. The \emph{mode} defines the behavior
of the parser with respect to the input. It's different behavior can be of the
type: accepting multiple declaration, skipping errors, accept/reject abstract
declarators, accept/reject implicit declarators etc... (see CPARSE\_MODE\_* for
full definition).

\begin{adjustbox}{bgcolor=backgroundColour, max width=\textwidth}
\begin{lstlisting}[style=CStyle]
typedef struct CPState {
  CPChar c;               /* Current character. */
  CPToken tok;            /* Current token. */
  CPValue val;            /* Token value. */
  GCstr *str;             /* Interned string of identifier/keyword. */
  CType *ct;              /* C type table entry. */
  const char *p;          /* Current position in input buffer. */
  SBuf sb;                /* String buffer for tokens. */
  lua_State *L;           /* Lua state. */
  CTState *cts;           /* C type state. */
  TValue *param;          /* C type parameters. ($xyz)*/
  const char *srcname;    /* Current source name. */
  BCLine linenumber;      /* Input line counter. */
  int depth;              /* Recursive declaration depth. */
  uint32_t tmask;         /* Type mask for next identifier. */
  uint32_t mode;          /* C parser mode. */
  uint8_t packstack[...]; /* Stack for pack pragmas. */
  uint8_t curpack;        /* Current position in pack pragma stack. */
} CPState;
\end{lstlisting}
\end{adjustbox}

\paratitle{lj\_cparse.c:}\\
This file contains the code of a simple lexer and a simplified, not valid c
parser. It uses the \emph{CPState} for the parsing of the input
string and the \emph{CPDecl} structure below for the construction of the
corresponding CType. During parsing the chain of typdef is unrolled (typdef are
still internalized for future reference but are not chained to the created
ctype.)

\begin{lstlisting}[style=CStyle]
typedef struct CPDecl {
  CPDeclIdx top;     /* Top of declaration stack. */
  CPDeclIdx pos;     /* Insertion position in declaration chain. */
  CPDeclIdx specpos; /* Saved position for declaration specifier. */
  uint32_t mode;     /* Declarator mode (same as CPState) */
  CPState *cp;       /* C parser state. */
  GCstr *name;       /* Name of declared identifier (if direct). */
  GCstr *redir;      /* Redirected symbol name. */
  CTypeID nameid;    /* Existing typedef for declared identifier. */
  CTInfo attr;       /* Attributes. */
  CTInfo fattr;      /* Function attributes. */
  CTInfo specattr;   /* Saved attributes. */
  CTInfo specfattr;  /* Saved function attributes. */
  CTSize bits;       /* Field size in bits (see Ctype bsz). */
  CType stack[...];  /* Type declaration stack. */
} CPDecl;
\end{lstlisting}

\paratitle{lj\_cdata.c:}\\
This file contains the functions responsible for doing cdata management, such as
allocations, free, finilizer, getter, setter and indexing.

\paratitle{lj\_cconv.c:}\\
This file is responsible for ctype conversion. It is divided in 5 parts: C type
compatibility checks, C type to C type conversion, C type to TValue conversion
(from C to Lua : i.e returned values), TValue to C type conversion (from Lua to
c: i.e passed arguments), and initializing C type with TValues (Initialization
of struct/union/array with Lua object)

\paratitle{lj\_carith.c:}\\%
This file contains the implementation far all built-in cdata arithmetic, such
as pointers arithmetic and integer arithmetic. It mainly manipulate some
\emph{CDArith} structure shown below.
\begin{lstlisting}[style=CStyle]
typedef struct CDArith {
  uint8_t *p[2]; /* data  of the two operands */
  CType  *ct[2]; /* ctype of the two operands */
} CDArith;
\end{lstlisting}

\paratitle{lj\_ccall.c:}\\
This file contains the code handling calls to C function. It does struct/array
register classification (see CCALL\_RCL\_*), computing how it can be passed
as argument/return values (in gp register, sse register or memory). It handles
then the decomposition/packing depending of the calling convention picked.

\paratitle{lj\_ccall.h:}\\
Below is the main structure use for C function call.
\begin{lstlisting}[style=CStyle]
typedef struct CCallState {
  void (*func)(void); /* Pointer to called function. */
  uint32_t spadj;     /* Stack pointer adjustment. */
  uint8_t nsp;        /* Number of stack slots. */
  uint8_t retref;     /* Return value by reference. */
  uint8_t ngpr;       /* Number of arguments in GPRs. */
  uint8_t nfpr;       /* Number of arguments in FPRs. */
  [...]
  FPRArg fpr[...];    /* Arguments/results in FPRs. (SSE) */
  GPRArg gpr[...];    /* Arguments/results in GPRs. */
  GPRArg stack[...];  /* Stack slots. */
} CCallState;
\end{lstlisting}

\paratitle{lj\_ccallback.c:}\\
This file contains the FFI C callback handling. The principal structure is the
\emph{CCallback} below (see \emph{lj\_ctype.h}) mainly use through the cb field
of the \emph{CTState}
structure. Each callback is associated with a unique cb slot and the
\emph{cts-$>$miscmap} contains the mapping between cb slot and function
pointers. The \emph{cts.cb.cbid} is a table mapping cb slot to the corresponding
CTypeID. \emph{cts.cb.mcode} is an mmap'ed executable page that contains a push
of slot id and a jump to \emph{lj\_vm\_ffi\_callback} (the entry in this page
corresponds to the cb address provided to the C code). Below is a description
of the call order when a callback is called.

\begin{itemize}
	\item C call a callback address
	\item arrive in the callback mcode page (push the appropriate slot id and call lj\_vm\_ffi\_callback)
	\item arrive in lj\_vm\_ffi\_callback that charge in registers the callback.fpr/gpr + stack in registers
	\item arrive in lj\_ccallback\_enter that prepare the Lua state and do conversion of argument from C to Lua type.
	\item arrive in lj\_vm\_ffi\_callback : execute the callback with this Lua state
	\item Lua callback
	\item arrive in $|->$cont\_ffi\_callback: call lj\_ccallback\_leave
	\item arrive in lj\_ccallback\_leave convert return value from Lua to C.
	\item arrive in $|->$cont\_ffi\_callback: return to C code with the return value in register/stack.
	\item arrive in C code.
\end{itemize}

\begin{lstlisting}[style=CStyle]
typedef struct CCallback {
  FPRCBArg fpr[...]; /* Arguments/results in FPRs. */
  intptr_t gpr[...]; /* Arguments/results in GPRs. */
  intptr_t *stack;   /* Pointer to arguments on stack. */
  void *mcode;       /* Machine code for callback func. pointers. */
  CTypeID1 *cbid;    /* Callback type table. */
  MSize sizeid;      /* Size of callback type table. */
  MSize topid;       /* Highest unused callback type table slot. */
  MSize slot;        /* Current callback slot. */
} CCallback;
\end{lstlisting}

\paratitle{lj\_clib.c:}\\
This file contains the necessary code to load/unload ffi library. It also handles
the indexing of the external library using named symbol. It uses platform
specific tools to explore the exposed symbols. Every symbol is resolved only once
and cached in the \emph{CLibrary} cache table.

\paratitle{lj\_clib.h:}\\
\begin{lstlisting}[style=CStyle]
typedef struct CLibrary {
  void *handle; /* Opaque handle for dynamic library loader. */
  GCtab *cache; /* Cache resolved symbols. Anchored in ud->env. */
} CLibrary;
\end{lstlisting}


    