\definecolor{mGreen}{rgb}{0,0.5,0}
\definecolor{mGray}{rgb}{0.95,0.95,0.95}
\definecolor{mPurple}{rgb}{0.58,0,0.82}
\definecolor{mViolet}{rgb}{0.6,0.19,0.8}
\definecolor{mBlue}{rgb}{0.36,0.44,0.92}
\definecolor{backgroundColour}{rgb}{1,1,1}
\definecolor{mBlack}{rgb}{0,0,0}


\lstdefinestyle{LuaStyle}
{
  %backgroundcolor   = \color{white},
  commentstyle      = \color{mGreen},
  keywordstyle      = \color{magenta},
  numberstyle       = \tiny\color{gray},
  stringstyle       = \color{mPurple},
  basicstyle        = \scriptsize\ttfamily,
  captionpos        = b,
  keepspaces        = true,
  numbers           = left,
  numbersep         = 9pt,
  showspaces        = false,
  showstringspaces  = false,
  showtabs          = false,
  tabsize           = 2,
  language          = {[5.2]Lua},
}
\lstset{style=LuaStyle}

\mdfdefinestyle{LuaStyleFrame}{
    backgroundcolor=black!5,
    linecolor=white,
    innertopmargin=0.5pt,
    innerbottommargin=0.5pt,
    innerrightmargin=4pt,
    innerleftmargin=4pt,
    leftmargin = 4pt,
    rightmargin = 4pt,
}

\mdfdefinestyle{DumpStyleFrame}{
    backgroundcolor=white,
    linecolor=white,
    innertopmargin=0.5pt,
    innerbottommargin=0.5pt,
    innerrightmargin=4pt,
    innerleftmargin=4pt,
    leftmargin = 4pt,
    rightmargin = 4pt,
}

\lstdefinestyle{CStyle}{
    backgroundcolor   = \color{backgroundColour},
    commentstyle      = \color{mGreen},
    keywordstyle      = \color{magenta},
    numberstyle       = \tiny\color{mGray},
    stringstyle       = \color{mPurple},
    basicstyle        = \scriptsize\ttfamily, 
    % breakatwhitespace = false,
    % breaklines        = true,
    captionpos        = b,
    keepspaces        = true,
    numbers           = left,
    numbersep         = 5pt,
    showspaces        = false,
    showstringspaces  = false,
    showtabs          = false,
    tabsize           = 2,
    language          = C,
}

\lstdefinestyle{DumpStyle}{
    %backgroundcolor = \color{white},
    stringstyle       = \color{mGreen},
    basicstyle        = \scriptsize\ttfamily\color{black},
    breaklines=false,
    numbers=none,
    language = ir,
}

\lstdefinelanguage{ir}
{
keywords=[1]{num},
keywords=[2]{tab},
keywords=[3]{int},
keywords=[4]{NULL},
keywordstyle=[1]\color{mBlue},
keywordstyle=[2]\color{red},
keywordstyle=[3]\color{mViolet},
keywordstyle=[4]\color{red},
sensitive=false,
morestring=[b]",
}

\lstset{numbers=none, style=DumpStyle}


\tikzstyle{squarednode} = [rectangle, line width=1pt, draw=black, fill=black!5, minimum width=2cm, align=left]
\tikzstyle{emptynode} = [draw=none, fill=none]
\tikzstyle{frame} = [line width=1pt, draw=gray, inner sep=7mm, minimum width=4mm,minimum height=5mm]
\tikzstyle{frame_loop} = [line width=1pt, draw=gray, inner sep=5mm, minimum width=1mm,minimum height=1mm]
\tikzset{font={\fontsize{9pt}{12}\ttfamily}}

% Multicolumns
\setlength{\columnsep}{1cm}

% customized commands
\newcommand{\keyword}[1]{\textcolor{red}{\emph{#1}}}
\newcommand{\myblocknode}[2]{\node[squarednode,label={[shift={(0.0,0)}]\scriptsize[#1] \hspace{0.5cm} #2}] }
\newcommand{\myloopframe}[2]{\node[label={[shift={(0,0)}]LOOP \hspace{1.3cm} \scriptsize0001}, frame_loop,fit= (#1) (#2)]{}}
\newcommand{\mylooplabel}[1]{LOOP \hspace{1.3cm} \scriptsize#1}
\newcommand{\mymargin}[1]{}

% shift -0.4 \hspace{0.3cm} (0001)

% To remember
% Command to shift the label: \node[label={[shift={(-1.2,0)}]name},...]
% Command to make a curve line without control: \draw[->] (node9.east) to[out=0,in=90, distance=5cm] (node0.north);
% Command to make straight lines (see example tikz) \draw[|->] (node6.south) --++  (-3,-1) |- (node4.west) node[near start, above] {\tiny{T}};